\chapter*{Résumé}
\addcontentsline{toc}{chapter}{Résumé} 

\begin{mdframed}
\vspace{-.25cm}
\paragraph*{Titre:} Conception et évaluation de politiques publiques dans le cadre des retraites.

\begin{small}
\vspace{-.25cm}
\paragraph*{Mots clefs:} Réformes des retraites ; Portabilité des droits ; Travailleurs indépendants ; RATP ; Assiette sociale ; Cotisations sociales

\vspace{-.5cm}
\setlength{\columnsep}{12pt} % I want the columnsep to be wider only on this page.
\begin{multicols}{2}
\paragraph*{Résumé:} 
Ce mémoire traite de deux grandes réformes liées aux politiques publiques dans le cadre des retraites en France : l'ouverture à la concurrence des lignes de bus RATP et la réforme de l’assiette sociale des travailleurs indépendants (TI) et des professions libérales (PL). La première réforme, initiée par le droit européen, vise à garantir la portabilité des droits à la retraite des salariés transférés à de nouveaux employeurs via un "sac à dos social". Ce dispositif assure la continuité des droits sociaux des agents RATP lors de la transition vers des entreprises privées.

La seconde réforme concerne la simplification du calcul des cotisations sociales des TI et PL, en fusionnant les assiettes de cotisations et de contributions sociales (CSG et CRDS). Cela vise à réduire les inégalités entre ces professions et les salariés, tout en améliorant la transparence du système et l’acquisition de droits sociaux.

Ces deux réformes, bien que distinctes, illustrent les défis actuels d'équité, de simplification administrative et de durabilité des régimes de retraite en France, enjeux majeurs pour les politiques publiques.

\end{multicols}
\end{small}
\end{mdframed}

\clearpage
